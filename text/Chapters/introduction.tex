\chapter{Úvod}
\label{sec:Introduction}
Počítání je jednou ze základních lidských dovedností. Formálně se jedná o proces, kdy je pro konečnou množinu určen počet jejích prvků, a je to tak fundamentální činnost, že se ji lidé učí již ve velmi útlém věku.
Archeologické nálezy, jako například vlčí kost s řadou zářezů objevená v roce 1936 v Dolních Věstonicích \cite{vestonice}, ukazují, že lidé počítali již před desítkami tisíc let.
Od té doby se počítání stalo tak velkou součástí každodenního života, že už jej lidé berou jako naprostou samozřejmost a počítají věci mnohokrát za den, aniž by o tom jakkoliv přemýšleli.

Není proto ku podivu, že problematika počítání je důležitá i v oblasti strojového vidění, kde pro ni existuje celá řada aplikací, kde by se dala využít.
Často totiž chceme proces počítání automatizovat, a to hlavně v případech kdy je dataset, ve kterém jsou objekty počítány, příliš obsáhlý a využití lidské síly by v takové situaci bylo neefektivní.
Strojové vidění v takovém případě nabízí relativně přesnou, rychlou a přitom neintrusivní metodu, jak objekty spočítat.

Počítání objektů v obraze může například sloužit v řízení dopravy pro zjišťování zaplněnosti parkoviště či určení hustoty dopravy na silnici. Také je možné jej využít třeba v ekologických statistikách pro odhady velikosti hejn ryb či ptáků.
Tato práce se ale specificky zabývá počítám lidí v obraze, které má také nespočet potenciálních využití.
Za zmínění rozhodně stojí použití v oblasti Smart Cities, kde informace o hustotě a počtu lidí může mít význam například v plánování hromadné dopravy nebo řízení přechodů.
Také je ale nutné zmínit, že tato technologie by mohla být použita pro masové monitorování chování lidí.

Cílem této práce je seznámit se s principy používanými pro počítání lidí v obraze a následně na základě těchto informací navrhnout a implementovat program, jenž bude počítat lidi v jednotlivých snímcích vstupní videosekvence.
V první části této práce je krátce shrnuta historie této disciplíny. Následuje popis navrženého řešení a  výsledky získané testováním tohoto řešení na dostupných datasetech.


\endinput