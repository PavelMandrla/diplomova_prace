\chapter{Historie}
\label{sec:History}
Moderní metody počítání lidí v obraze využívají neuronové sítě, avšak experimenty v této přišly již dlouho před tím, než se staly neuronové sítě populární.
Pro spočítání osob v obraze tedy tehdy musely být využity jiné metody.
Obecně se metody používané v této oblasti dají rozdělit do dvou kategorií a to na přímé (direct) a nepřímé (indirect). \cite{crowd_on_pets}

Při použití přímých metod je použita nějaká forma segmentace obrazu a následně jsou jednotliví lidé v obraze detekováni a spočítáni. Do této kategorie by se dal zařadit například 

 TODO

Problémem přímých metod byla obtížná detekce jednotlivých lidí v hustě zalidněných scénách.
V davech totiž často dochází k částečnému zakrývání jednotlivých lidí, jinými lidmi, což dělá proces segmentace a následné klasifikace velmi složitým.
Nepřímé metody počítání lidí se tento problém snaží vyřešit tím, že pro stanovení počtu lidí ve scéně 
měří veličiny, které nejsou založeny na znalosti pozice každého člověka ve scéně.

Conte, Foggia, Percannella et al. ve svém článku \cite{crowd_on_pets} vytvořili estimátor, jenž určoval počet lidí na základě počtu a hustoty SURF příznaků nacházejících se v popředí snímku.
Tento postup je založen na myšlence, že v hustém davu, kde bude docházet k mnoha okluzím, se bude vyskytovat velké množství zájmových bodů, zatímco v případě, kdy jsou lidé ve snímku od sebe izolováni, bude v dané oblasti SURF bodů méně.
O tom, zda příznak patří do popředí, je rozhodnuto na základě vektoru popisujícího pohyb daného zájmového bodu mezi dvěma po sobě jdoucími snímky.
Autoři předpokládají, že i statičtí lidé se mírně pohybují, a proto všechny zájmové body s délkou pohybového vektoru větší, než je nějaký stanovený práh, jsou označeny, že patří lidem.
Přesnost tohoto detektoru proto bude velice snadno ovlivněna výskytem jiných pohybujících se objektů, které se mohou v obraze nacházet jako například automobily.
Autoři článku se také snaží odstranit negativní účinky perspektivy na přesnost výsledku, a to tím, že blízké zájmové body shlukují do skupin, které ohodnocují zvlášť.
Výslednou funkci, která pro shluk vrátí počet lidí, kteří se v něm nacházejí, je získaná pomocí SVR a je závislá na počtu bodů ve shluku, jejich hustotě a velikosti evaluovaného shluku.


S narůstající popularitou neuronových sítí v oblasti strojového vidění přirozeně narůstá i množství estimátorů počítajících objekty v obraze, které jsou na neuronových sítích založeny.
Například autoři článku \cite{YOLO_counting} pro tento účel upravili detekční síť YOLO.
Jejich úpravy této sítě spočívaly hlavně v rozdělení obrazu do více buněk a ve větším počtu bounding boxů počítaných pro každou buňku v obraze. Výslednou síť natrénovali tak, ať je citlivá na lidi.
Počet lidí v obraze byl stanoven jako počet bounding boxů, které se nacházely v oblasti, kterou v obraze vytyčili.






\endinput