\chapter{Navržené řešení}
\label{sec:Propesed_solution}

Velká většina výzkumných prací, které se počítáním lidí v obraze zabývají, se snaží jejich množství určit na základě jediného snímku.
Toto je celkem logicky dáno jednodušší dostupností anotovaných datasetů, které jsou v případě, že se jedná pouze o jednotlivé snímky a ne videosekvence, mnohem obsáhlejší a zachycují mnohem větší množství nejrůznějších situací z mnoha úhlů pohledu.
Temporální informace ale mohou být při počítání objektů v obraze velmi cennou informací, kterou by neuronová síť mohla využít pro snazší řešení situací, kdy jsou lidé v obraze zakryti jinými lidmi či objekty, a zlepšit tak přesnost výsledného počtu.
Je totiž velmi nepravděpodobné, že člověk procházející zachycovanou scénou mezi dvěma po sobě jdoucími snímky zničehonic ze scény zmizí.
Je samozřejmě možné, že tento člověk například zašel za roh, nebo byl zakryt kolem projíždějící dodávkou, ale pokud je stále alespoň trochu patrné, že se v obraze nachází a je překryt pouze částečně, tak by měl být do výsledného počtu rozhodně započítán. Pokud by taková situace byla vyhodnocována pouze na základě jediného snímku, tak by nepochybně bylo přesné spočítání lidí obtížné i pro člověka. Předchozí snímky přitom často obsahují informace, které by při rozhodování, zda se opravdu jedná o člověka, mohly výrazně pomoci.

\begin{figure}[h!]
	\centering
	\includegraphics[width=\textwidth]{Figures/solution/net_structure.pdf}
	\caption{Struktura navržené sítě}
	\label{fig:proposed_net}
\end{figure}

Neuronová sít navržená v této práci se tohoto snaží využít a jejím cílem je vytvořit ze vstupní sekvence \(n\) po sobě jdoucích snímků vytvořit dvoudimenzionální hustotní mapu, která vyobrazuje hustotu výskytu lidí v posledním snímku vstupní sekvence, a suma jejíž pixelů bude rovna celkovému počtu lidí v něm. Na obrázku \ref{fig:proposed_net} je vyobrazena struktura stíě navržené v této práci.
Jak je z obrázku patrné, tak se skládá z několika částí, které budou v této kapitole blíže popsány.

\section{ResNet}
Residuální neuronová síť (ResNet) \cite{ResNet} byla poprvé představena v roce 2015, kdy vyhrála ImageNet Large Scale Visual Recognition Challenge, a svou architekturou silně ovlivnila budoucnost návrhu neuronových sítí.
Síla této neuronové sítě spočívá v tom, že přináší způsob, jak snížit vliv tzv. problému mizejícího gradientu (vanishing gradient problem).
Komplexita neuronových sítí se s každým rokem zvyšuje. Nejvíce se ale konvoluční neuronové sítě rozrůstají do hloubky. Čím je totiž neuronová síť hlubší, tím komplexnější příznaky dokáže extrahovat z původního vstupu a tím složitější funkce dokáže aproximovat.
Čím je ale neuronová síť hlubší, tím více se projevuje problém mizejícího gradientu.
Když je při trénování sítí šířena chyba pomocí algoritmu backpropagation, dochází k tomu, že její vliv je na svrchních vrstvách menší, než na hlubokých. Je-li tedy síť příliš hluboká, tak gradient je na svrchních vrstvách tak malý, že nedochází k přenastavení vah a tyto vrstvy se tedy nijak nemění.
Prohloubení sítě tedy v tomto případě ztrácí smysl a dokonce může mít negativní vliv na její výsledky.
Mělčí neuronové sítě proto mohou při řešení stejného problému dosáhnout mnohem lepších výsledků.

\begin{figure}[h!]
	\centering
	\includegraphics[width=0.7\textwidth]{Figures/solution/vanishing_gradient.png}
	\caption{Porovnání chyby při testování a trénování dvou sítí o hloubce 20 a 56 vrstev určených ke klasifikaci na datasetu CIFAR-10 \cite{ResNet}}
\end{figure}

Síť ResNet je tvořena reziduálními bloky, které jsou podobné VGG blokům \cite{VGG}.
Každý blok obsahuje dvě konvoluce o velikosti 3x3 a každá konvoluce je následovaná normalizací dávky (batch normalization) a aktivační funkcí ReLU.
Co ale reziduální blok přidává je tzv. zkratkové spojení (shortcut connection), které právě pomáhá snižovat vliv mizejícího gradientu.
Zkratkové spojení vezme vstupní hodnoty reziduálního bloku a přidá je k hodnotám před spuštěním druhé aktivační funkce.
V praxi to znamená, že vstupní hodnoty prochází v bloku dvěma cestami, kdy v jedné jsou změněny dvěma konvolucemi, zatímco v druhé se těmto konvolucím vyhnou. Při zpětném šíření chyby sítí se tedy chyba dostane do horních vrstev mnohem rychleji, takže navrhovaná síť může být mnohem hlubší.

\begin{figure}[h!]
	\centering
	\includegraphics[width=0.3\textwidth]{Figures/solution/residual_block.pdf}
	\caption{struktura reziduálního bloku}
	\label{fig:residual_block}
\end{figure}

Síť ResNet je v navržené síti použita hned na vstupu a jejím cílem je zredukovat dimenzi vstupní sekvence po sobě jdoucích snímků a extrahovat z jednotlivých snímků důležité příznaky, které následně budou využity pro extrakci temporálních informací a konstrukci výsledné hustotní mapy.
Konkrétně je použita síť ResNet18, která obsahuje pouze čtyři reziduální bloky.
Tato varianta byla zvolena kvůli omezenému množství paměti na testovacím stroji.


\endinput