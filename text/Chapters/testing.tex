\chapter{Testování a výsledky}
\label{sec:Testing}

Navržená neuronová síť byla testována na sestavě s Linuxovou distribucí Ubuntu 21.10 a obsahující osmijádrový procesor AMD Ryzen 7 5800H, 16 GB RAM a grafickou kartu NVIDIA GeForce RTX 3060 6GB GDDR6.
Právě velikost grafické paměti se při testování sítě ukázala jako největší limitace.
Při akceleraci operací neuronové sítě pomocí grafické karty jsou model i právě zpracovávané data nahrány v grafické paměti, kde za běhu aplikace zabírají kolem pěti GB paměti. 
Na grafické kartě tak zůstávaly řádově stovky MB volné paměti, která bývá navíc z části zabraná systémovými aplikacemi, což nedávalo mnoho místa pro větší zesložitění architektury sítě.

Hlavním cílem testování bylo zjistit, jaký vliv má použití sekvence více snímků na vstupu na přesnost výsledného počtu.
Z toho důvodu bylo natrénováno několik modelů, které se lišily hodnotou parametru \texttt{sequence\_length} udávajícího délku vstupní sekvence, ze které model estimuje počet lidí v jejím posledním obraze.
Stejnou délku měly i sekvence v trénovací množině pomocí kterých byl estimátor natrénován.

Aby se lidskému mozku videosekvence zdály plynulé, jsou při snímání kamerou snímkovány s poměrně vysokou frekvencí. 
Ta může u klasického videa dosahovat například dosahovat hodnoty třicet snímků za sekundu, což znamená, že časový rozdíl mezi jednotlivými snímky je zhruba 0,03 sekundy.
Lidé jsou navíc poměrně pomalu se pohybující stvoření.
Člověk, který jde rychlostí \(4 kmh^{-1}\) se tak za třicetinu sekundy pohne o \(3,5 cm\).
Efekt tohoto pohybu v obraze také bude velmi záviset na tom, jak daleko od kamery se daná osoba nachází. 
V případě, že by scéna byla snímána kamerou s horizontálním rozlišením 1080 pixelů a s objektivem s ekvivalentní ohniskovou vzdáleností \(50 mm\), by člověk jdoucí kolmo ke směru pohledu kamery a vzdálený od ní 10 metrů urazil za třicetinu sekundy vzdálenost ekvivalentní 2,3 pixelů.
Je proto zřejmé, že rozdíly mezi dvěma po sobě jdoucími snímky proto budou minimální a bude v sekvenci tak bude obsaženo velké množství redundantních informací.
Z tohoto důvou byl u trénovaných sítí měněn i parametr \texttt{stride}, který udává krok mezi snímky v sekvenci.
Je-li hodnota tohoto parametru rovna jedné, nebudou mezi snímky v sekvenci žádné mezery a budou brány tak, jak se ve vstupním videu nacházejí.
V případě, že má tento parametr hodnotu dva, tak již bude jeden snímek přeskočen a v sekvenci se tak bude nacházet každý druhý snímek ze vstupního videa.






\endinput