\chapter{Úvod}
\label{sec:Introduction}
Počítání je jednou ze základních lidských dovedností.
Archeologické nálezy, jako například vlčí kost s řadou zářezů objevená v roce 1936 v Dolních Věstonicích, ukazují, že lidé počítali již před desítkami tisíc let a je to fundamentální činnost, kterou se lidé učí ve velice útlém věku a provádí ji mnohokrát za den, aniž by nad ní jakkoliv přemýšleli.

Není proto ku podivu, že problematika počítání je důležitá i v oblasti strojového vidění a existuje pro ni v tomto oboru celá řada aplikací, kde by se dala využít.
Počítání objektů v obraze může například sloužit v řízení dopravy pro zjišťování zaplněnosti parkoviště či určení hustoty dopravy na silnici. Také je možné jej využít třeba v ekologii pro pro odhad velikosti hejn ryb či ptáků.
Samotné počítání lidí v obraze my mohlo mít využití například v oblasti Smart Cities, kde informace o hustotě a počtu lidí může mít význam například v plánování hromadné dopravy nebo řízení přechodů.

Přestože je počítání pro člověka tak základní činností, tak u strojového vidění a umělé inteligence je 

\endinput